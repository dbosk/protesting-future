\section{During a Protest}
\label{DuringProtest}

During a protest, organizers and demonstrators might want to communicate, 
either among themselves or to the outside world.
The communication to the outside world can have at least two purposes.
The first one is simply to try to get more people to come to the demonstration.
The second is when a demonstrator wants to store something for posterity.
This can be a photo capturing police brutality or a part of 
a proof-of-demonstration (as in \cref{AfterProtest}).
There are a few problems that must be considered for communication during 
a protest.

If the participants use the phone network, i.e.\ infrastructure which is 
generally controlled by the government, participants can be tracked and bound 
to the location through their phone.
Thus using the phone network for communication is not an option if there is 
a risk for tracking and reprimands.
If they communicate over the phone network they can still use the techniques 
outlined in \cref{BeforeProtest}, but this only protects the communication --- 
it does not prevent the government from learning that a specific phone was 
present.

If they do not want to be tracked, there are two options:
\begin{enumerate}
  \item they must use another network infrastructure that is not controlled by 
    the government,
  \item the mechanisms in \cref{BeforeProtest,AfterProtest} must allow 
    executions without access to a global communications infrastructure during 
    the demonstration, i.e.\ they must be asynchronous.
\end{enumerate}
There are solutions to the first options: mobile ad-hoc networks.
The area of ad-hoc networks is far too wide for us to convey more than the 
general idea of the field in this chapter, we refer the reader to e.g.\ 
\textcite{AdHocNetworksBook} for details.
The idea of ad-hoc networks is to form a network using ad-hoc connections.
For example, if Alice can communicate with Bob, Bob in turn can communicate 
with both Alice and Carol, then Alice can communicate with Carol through Bob.
Protesters can use this technique to form an ad-hoc network at the physical 
location of the demonstration, thus avoiding the government-controlled 
telephone network.
Depending on the reach of the ad-hoc network, participants might get access to 
the global Internet through some node in the network.
If not, they are limited to communicating only between themselves.


For the second option listed above, we can achieve local communication, e.g.\ 
pairwise communication through bluetooth or \ac{NFC}.
However, communicating certain data to the outside world will still be the 
ultimate goal, but with this option there is at least the possibility to use 
the local communication possibilities in \cref{AfterProtest}.

