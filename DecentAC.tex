\citet{PPACforPubFS} analysed two dichotomous models of communication.
The first was the pull model, where a sender writes the message to a previously 
determined place.
When the recipients later check for new messages they go to the place and look 
for new messages.
A suitable analogy would be that of magazines published by sales in kiosks: 
people go to the kiosk to get the latest publications.
The second model was the push model, here a sender sends the message directly 
to the recipients.
Thus it is more like magazine subscriptions: the next issue arrives in the 
mailbox shortly after publication.
This is the model of the communication described in \cref{GroupProperties}.
\citeauthor{PPACforPubFS} found that achieving privacy in the pull model is 
easier than in the push model.
In fact, achieving privacy in the push model is very difficult.

We can start by looking at the pull model for communication.

Now to the push model, \dots
