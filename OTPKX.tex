\citet{OTPKX} argue that if the adversary controls the entire network, then the
approach to deniability taken by \ac{OTR} and Signal does not suffice.
The problem is that the adversary can record a transcript of all communications
that have taken place.
We know that the \ac{NSA} did exactly that~\cite{XKeyscore}, and specifically 
saved ciphertexts for later when the decryption key might be available.
%~\cite{NSAsavesCiphertexts}.
In this setting it does not matter if anyone can generate a false transcript of
a conversation between Alice and Bob, the regime knows exactly what Alice has 
sent and Bob received and vice versa.
The argument of \ac{OTR}-like schemes is that Alice and Bob have the possibility 
to deny anything about the conversation since it cannot be decrypted.

There are more than one way to approach this problem.
The first approach would be to use an anonymizing service, such as 
Tor~\cite{Tor}.
This way, the regime would not know that Alice communicates with Bob, only that
Alice communicates with someone.
However, for all low-latency solutions, when the entry point and exit from the 
anonymizing network are both controlled by the adversary, then the adversary 
can perform a correlation attack and essentially render the anonymization 
service useless~\cite{SystemsForAnonymousCommunication}.
This is in fact the case if the regime controls the nation-wide network while 
critics of the regime, all located in the country, want to communicate in 
real-time.
To make this attack more difficult for the regime's surveillance
agency, the system must 
introduce random delays in our communication. %TODO: explain why
And despite all this, the regime can still ask Alice to decrypt the 
conversations --- either she complies or claims she do not know the key.

The second approach would be to ensure deniability even against this strong 
adversary.
This would not hide who communicates with whom, as in our first approach, but 
it provides deniability for the conversations.
The scheme suggested by \citet{OTPKX} makes use of one practical instance of 
deniable encryption~\cite{DeniableEncryption}.
They construct a scheme where Alice and Bob can create \enquote{false 
witnesses} for their conversation.
Basically Alice can create a decryption key such that when used to decrypt the 
ciphertext recorded by the regime from the network it will decrypt to 
a plaintext of Alice's choice.
This way she can \enquote{prove} her innocence.
However, the question whether the regime would actually accept such 
a \enquote{proof}, knowing it can equally well be false, remains open.
