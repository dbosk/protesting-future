\section{Before a Protest}
\label{BeforeProtest}

We are interested in solving several problems before a protest.
Essentially, the goal in this stage is to organize a protest, once we have 
solved this problem we can move on to having a protest.
Organizing a protest introduces several problems that must be solved.
Assume that Alice wants to organize a protest, then she must solve at least the 
following problems:
\begin{enumerate}
  \item Alice must find other interested people, i.e.\ potential fellow 
    organizers.
  \item The organizers must be able to communicate securely with each other.
  \item Alice must \enquote{spread the word} to other interested people, i.e.\ 
    potential participants.
  \item Finally, Alice and everyone interested must agree on a time and a place
    to have the protest.
\end{enumerate}

Finding other people who have similar interests can be difficult in general.
It is even more difficult if the interests are not socially accepted, e.g.\ 
being HBTQ or wanting change in a totalitarian regime.
These interests can be punished by death in several countries.
Thus Alice is very reluctant to reveal her interests in organizing a protest 
against the regime, the consequences of talking to the wrong person can be 
severe.
There are technical solutions which can make this easier, we will discuss these 
in \cref{UserSearch}.

Once Alice has found some people who has expressed interest she can proceed to 
the next problem.
We say \enquote{expressed interest} since there is no way for Alice to ensure 
this, a person can always lie about their interests to Alice.
This brings us to Alice's privacy expectations for communication.
The organizers do not want the regime's intelligence services to identify them 
as such.
We will discuss two-party conversations in \cref{Communicating}.
However, sometimes we desire a discussion between more than two participants at 
a time.
We will focus on this in \cref{Discussions}.

Eventually it is time for the protest.
None of the organizers and none of the participants want to show up for 
a demonstration alone or with just a few others.
This would make them an easy prey for the regime.
We will discuss the different aspects of scheduling the event in 
\cref{Scheduling}, i.e.\ choosing a time and a place, committing to participate 
etc.

Finally, many of these techniques requires the user to authenticate in some 
way.
This means that a user must keep their credentials.
This can be done by storing keys in a device, such as a smartphone.
However, a device can be lost and, as was lately shown~\cite{AppleVsFBI}, can 
easily attract attacks.
We will discuss the aspects of how to protect \enquote{accounts} and data in 
the decentralized setting in \cref{passwd}.

\subsection{Searching for Your Friends}
\label{UserSearch}

So the challenge is to protect user data from malicious adversaries but
at the same time making users findable for other legitimate users.
To distinguish between these two cases, we assume that legitimate users
possess more information about a target user than the adversary.
Then a knowledge threshold can be enforced using cryptographic techniques, to
guarantee that a user can only be found if the party searching for her
can present enough details about her (\enquote{find me if you know enough about
me}).

Two protocols' implementations are presented by \citet{ThresholdUserSearch} 
that have different advantages and disadvantages.
Neither rely on any central repository of user data but are suitable to be 
implemented in a completely decentralized way using \iac{DHT}.
This avoids the biggest risk to user data: the leakage of a central database 
with sensitive information about a large number of people.

The proposed protocols allow users to register their identifiers (e.g.\ 
links to their profile pages, e-mail addresses or other contact
information) and specify the required knowledge that is needed to find
this information (e.g.\  name, city, workplace and date of birth).
One implementation guarantees this knowledge-threshold by encoding the
storage location of the registered user identifiers using the required
knowledge attributes.
Only users that know these attributes can construct a valid lookup request for 
the \ac{DHT} that will return the desired user identifier.
The other protocol stores user identifiers encrypted in the \ac{DHT} and uses 
threshold secret-sharing techniques to guarantee that no user with less than 
the required number of attributes can decrypt a stored identifier.

Neither protocol can provide perfect protection.
In the worst-case of a targeted attack, an adversary with profound background 
knowledge about the target user will likely succeed.
For example, we cannot protect the user identifier if the adversary knows as 
many attributes about the target user as legitimate users do.
At the same time, both schemes protect the users fairly well from large-scale 
crawling attacks as the search space of all possible attribute combinations is 
too large to brute-force and the protocols transform the registered user data 
in such a way that inferences from the publicly stored data are infeasible.
Even if the adversary focuses her effort to only crawl the data of a specified 
subset of the user-base (e.g.\ all persons working at a specific organization), 
the proposed protocols offer good protection. 

The knowledge-threshold is an individual user parameter, so users that
consider themselves to be more exposed to risks can choose a higher
knowledge-threshold to increase their protection at the cost of a lower
usability, as a higher threshold makes it harder for other legitimate
users to find them.
In that sense, the presented protocols allow users to individually balance 
their findability and privacy requirements.



The Gossple scheme by \citet{Gossple} \dots


\subsection{Communication between People}
\label{Communicating}

We will now focus on the communication.
Specifically we will focus on communication between pairs of people, e.g.\ 
Alice talking to Bob.
\citeauthor{otr2004} designed a secure protocol for two-people communication, 
the \ac{OTR} protocol.
They desired an electronic equivalent of face-to-face conversations, i.e.\ that 
they leave no proofs of any kind behind:
if Alice and Bob has had a conversation, Bob cannot go to Eve afterwards and 
prove anything about what Alice has said --- the same as in a face-to-face 
conversation.
This property is not true for email or most centralized communication services.

\subsubsection{Standard Email}

The standard email system does not provide any security.
A suitable analogy would be that each message is a postcard, i.e.\ it has no 
envelope, so the content and address are visible on it.
This means that the postman can read the cards' contents, their recipients' and
senders' addresses.
(Yes, unlike real postcards these also include the sender's address.)
Furthermore, most postmen use transparent sacks to carry the postcards, so 
everyone along the way can also read the sender's and recipient's address and 
the contents.
However, some postmen have started using non-transparent sacks, i.e.\ encrypted 
connections between the servers, so those postcards can only be read by the 
staff in the post-office.
Thus the email system provides no confidentiality: each email server can read 
the messages, each network operator along the transport route can also read 
(and make a copy of) each email.
However, it is actually worse than that, because the email system provides no 
integrity either.
This means that the postman, or anyone along the way, can do arbitrary 
modifications to the messages without anyone noticing the difference.
We can safely say that we cannot rely on the email system for neither security 
nor privacy when planning a protest.

When using a centralized communications service, such as Facebook, the level of
security and privacy we can achieve is that the postman carries non-transparent
sacks.
The business model of most such services is to read peoples postcards to better
profile their interests and thus deliver better suiting advertising.
Here, third parties cannot directly see who is communicating with whom.
They can only see that something goes to and from the service.
However, all information is available internally to the service.
This means that there are ways of learning this, for example through 
PRISM~\cite{Prism} of the \ac{NSA}.

\subsubsection{Secure Email and Text Messaging}

Secure email works by employing cryptography: we encrypt the contents of the 
postcard, providing confidentiality, and then add a digital signature to 
prevent modifications.
Thus the recipient is the only one who can read the message and the recipient 
can also verify that the message has not been modified along the way.
To make key management easy, most schemes use public-key cryptography.
This means that we have two keys, one which is public and another which is kept
private.
For encryption, the public key can transform a message to a ciphertext, i.e.\ 
a random-looking text string.
The private key can be used to transform the ciphertext back to the message.
Given only the public key, it is \enquote{impossible} to find the private key.
For signatures, we can use the private key to compute a signature of a message 
and then send the message and its signature.
The recipient can then use the public key to verify the signature of the 
message.
This signature depends on the entire message, so it is impossible to move 
a signature to another message --- unlike signatures on paper.
And since it is impossible to find the private key given only the public key, 
no one can create fake signatures.

One problem with this approach to secure email is that the sender and recipient
are still in the clear, anyone can read them.
So the content is hidden, but the meta-data is not.

Another problem is that the digital signatures used provides a property called 
non-repudiation.
Say that Alice securely sent an email to Bob, if Eve would compromise Bob's 
private key, as many government agencies can, then she would learn that Alice 
--- and no one else --- has sent that message to Bob.
Bob might even give the message and his key to Eve voluntarily or under threat.
This is exactly the property that \citeauthor{otr2004} wanted to remove with 
\ac{OTR}.
They can do this by leveraging the interactive nature of \ac{IM} and changing 
the digital signatures to shared-key \acp{MAC}.
Shared-key means that Alice and Bob share the same key for generating and 
verifying \iac{MAC}.
This means that Bob can generate valid \acp{MAC} for any message and show to 
Eve, thus he cannot prove to Eve what Alice has said --- since he could have 
created this \enquote{proof} himself.
In addition, Alice and Bob do not use the same \ac{MAC} key throughout their 
conversation, then continuously exchange new keys, one for each message.
However, in this situation, Eve still has only two candidates as the author of 
the message: Alice and Bob, since they both have access to the shared keys.
To remedy this problem Alice and Bob publishes the \ac{MAC} keys after use, 
i.e.\ when they no longer need them.
This gives \enquote{everyone} the possibility of generating messages that 
verifies under Alice and Bob's key, so now Alice and Bob can argue that someone 
(Eve included) could have modified the ciphertext.

The \ac{OTR} protocol became widely spread after the 2013 revelations about the
mass surveillance of the \ac{NSA} and \ac{GCHQ}, many derivatives of the 
protocol emerged in smartphone apps.
Among the most wide-spread derivatives of \ac{OTR} is Signal (formerly 
TextSecure)~\cite{SignalApp}\footnote{%
  TextSecure actually existed before the Snowden revelations, but has seen more
  wide-spread use after.
}.
The Signal protocol has, unlike many other of the derivatives, been formally 
analysed and proven that it indeed provides its claimed security 
properties~\cite{TextSecureAnalysis}.
One improvement over \ac{OTR} is the deniability.
In Signal the authentication is set up in such a way that any person knowing 
the public key of Alice and Bob can generate a fake transcript of 
a conversation.
This results in that Eve has many more candidates for the authors of 
a conversation.

\subsubsection{When the Adversary Controls the Network}

\citet{otr2004} \dots

\citet{OTPKX} \dots




\subsection{Holding Discussions}
\label{Discussions}

\subsection{Holding Discussions}
\label{Discussions}

So far we have treated only one-to-one conversations, i.e.\ Alice and Bob 
talking to each other.
However, there are usually more than two people organizing a protest, and so we 
need to hold discussions with more than only two people at a time.
In this situation there are two approaches to solving the communication:
simultaneous pairwise (one-to-one) communication between all participants or 
multicast communication.
Furthermore, how the messages are distributed is also important, because Eve 
might be able to learn who the participants are.

\subsubsection{Group Communication}
\label{GroupCommunication}

When a group uses pairwise communication, every member of the group will set up 
a one-to-one channel to each other member of the group.
Each pairwise channel is as described above (\cref{Communicating}).
Then for every message Alice wants to send to the group she has to send it to 
every member.
This would allow Alice to cheat, e.g.\ she can send \enquote{Who wants to 
  overthrow the regime?} to everyone except to Bob, to whom she instead sends 
\enquote{Who wants to order pizza?}.
This opens up for the Byzantine Generals' problem~\cite{ByzantineGenerals}, 
where malicious actors can lie to honest actors to disrupt operation.
\textcite{ByzantineGenerals} in fact proved that it is impossible for the 
honest parties to recover from this and identify the malicious parties if the 
malicious parties exceed a third of the participants.
  
Although Alice's ability to say different things to different participants is 
in itself a desirable property from Alice's perspective --- she would like to 
lie to suspected regime agents --- this property can at the same time be 
undesirable due to the Byzantine Generals' problem.
For this reason group communication must provide better properties, namely that 
everyone hears who said what and when, thus forcing Alice to say the same thing 
to all participants.
In such a scheme, when Bob replies \enquote{I do, shall we say tonight?} the 
others will see that Bob is replying to something they did not see and not to 
the question \enquote{Who wants to overthrow the regime?}.

%\textcite{multiotr2009} tried to extend the \ac{OTR} protocol to a multi-party 
%setting.
%This did not result in a concrete protocol implementation, and the resulting 
%protocol they suggested was also very complex.
%It also had some undesirable limitations, for instance, the scenario that Bob 
%receives a question which is different from everyone else's is only detected at 
%the very end of the conversation.
%As is pointed out by \textcite{TSgroups}, asynchronous communication today has 
%no real end, which makes the approach of \citeauthor{multiotr2009} even less 
%appealing.
%Due to this,
\textcite{SignalApp} implements group chats as simple pairwise conversations.
With additional meta-data they can ensure consistent history~\cite{TSgroups}.
A technique that could be used for this is to include a message 
digest\footnote{%
  A message digest is computed using a cryptographic hash function, i.e.\ 
  a one-way function.
  The output of such functions is unpredictable, e.g.\ finding two inputs that 
  yield the same output is \enquote{impossible}.
  It is also \enquote{impossible} to infer its input from its output.
  By \enquote{impossible} we mean that the probability of success is close to 
  \(\sfrac{1}{\sqrt{2^n}}\), where \(n\) usually is at least \(256\).
} of the entire conversation history with each message.
If Alice would cheat Bob as above, this means that the message digest included 
in Bob's reply and the one computed by the other participants above would 
differ, thus everyone learns that the conversation history is inconsistent and 
should no longer be trusted.
Due to the unpredictable property of the message digest, Alice cannot phrase 
the two different messages in such a way that they yield the same message 
digest in the history either.
But despite this, the other participants cannot determine if it is Alice or Bob
who is lying about the message history --- Alice could send the same message to
everyone and still Bob could try to frame her.

\subsubsection{Message Distribution}
\label{MessageDistribution}

\textcite{PPACinPubFS} analysed two dichotomous models of communication: the 
pull model and the push model.
Alice wants to send a message to Bob and Carol.
In the pull model, Alice would publish her message in a place known to Bob and 
Carol.
Bob and Carol visits this place periodically to check if Alice has published 
any new messages.
(If there is a new message, they make a copy for themselves.)
In the push model, Alice drops her message in Bob's and Carol's letter boxes.
(One copy for each.)
Email and text-messaging are best modelled using the push model (see 
\cref{GroupCommunication}).
For both models, Eve can analyse the communication patterns to infer (a part 
of) the social graph unless Alice, Bob and Carol use some countermeasures.

%\citeauthor{PPACinPubFS} found that achieving privacy in the pull model is 
%technically easier than in the push model.
%In fact, achieving strong privacy in the push model is very
%difficult. %TODO: explain why (move to end)

\paragraph{The Pull Model}

We can start by looking at the pull model for communication.
Alice wants to distribute a message to Bob and Carol, the participants in 
a discussion.
In the pull model they actively ask Alice (or an intermediary) for new messages 
at regular intervals.
To form a protocol around this model, Alice, Bob and Carol can agree on 
a location where Alice puts her messages\footnote{%
  Technically, this can be implemented in a similar fashion as the \ac{DHT}, as 
  mentioned in \cref{UserSearch}.
  This would make it more difficult for Eve to censor it compared to the 
  centralized systems.
}.
When Alice wants to send a new message, she writes it to this particular 
location.
When Bob and Carol want to, they can read from the location to see if there are 
any new messages.

Suppose that Eve controls the network that Alice uses\footnote{%
  This is reasonable considering that we saw earlier that the \ac{NSA} and 
  \ac{GCHQ} achieve similar results in free countries.
}.
Since we have a decentralized system in mind, we can also assume that anyone 
(especially Eve) can read anything from the storage.
%This is why Alice encrypts all her messages for the recipients' keys.
%Also, Alice does not want to be associated with the message, not authorship nor 
%posting it.

The first thing we can say about this situation is that Alice would like to 
have confidentiality for the messages' contents, so that Eve cannot read her 
messages.
Alice would also like to have integrity for her messages, so that Bob and Carol 
can be sure that they are from Alice and that Eve has not modified them.
Many systems provide these two properties, e.g.\ \ac{PGP} does this for email 
(and could be applied here as well).
However, Alice also wants to hide the sender and recipients, which many systems 
(including \ac{PGP}) do not provide.
There is a class of encryption schemes called \ac{ANOBE} schemes.
This type of scheme provides confidentiality while hiding the sender and the 
intended recipients.
If Alice can write the message anonymously to the storage and the message is 
encrypted using \iac{ANOBE} scheme, then it will be difficult to determine the 
sender.
Furthermore, if the recipients fetch the messages anonymously too, then the 
recipients are also hidden.
The idea is as follows: if Eve cannot distinguish between Bob and Carol 
fetching a message, then it might just as well only be Bob who fetches messages 
from this location --- Eve cannot tell the difference.
(We will return to this problem later.)

The problem of integrity remains.
There are two approaches: digital signatures and \acp{MAC}.
If Alice, Bob and Carol agree on a commonly shared \ac{MAC} key, then they can 
use \acp{MAC} to ensure integrity.
One advantage of \acp{MAC} is that anyone who can verify the authenticity of 
\iac{MAC} can also create one (as was pointed out above).
With digital signatures on the other hand, if Alice signs a message it is clear 
that Alice is the only one who could have signed it.
(It is important that only Bob and Carol know that Alice owns the private key, 
and that it remains anonymous to Eve.)
But this provides Eve with something to track messages by, all messages signed 
by the same key are related.
With \acp{MAC}, Bob and Carol could also have authored the message and Eve 
cannot determine which messages are related either.
This means that for a discussion, any of the participants would be equally 
likely to be the author of a given message.
However, this relies on the anonymity of the actors.

\paragraph{The Push Model}

In the push model, Bob and Carol have one location each where Alice will drop 
her messages.
(She can achieve confidentiality and integrity similarly as in the pull model.)
One thing we can observe is that the recipients are not as hidden as in the 
pull model, even if we assume anonymity.
Eve can observe that someone (Alice, but Eve does not know that) puts two 
messages at the same time.
Eve can then observe these locations to see when someone (Bob or Carol, but Eve
does not know that either) reads messages from those locations.
The main problem with the push model is that it reveals more meta-information 
than the pull model does.
With the push model Eve can build herself a map of the social graph\footnote{%
  Since Eve works with probability distributions, this would be an 
  approximation of the social graph.
  But her approximation can come very close to the real one.
}.
Then she only needs to map the real identities of Alice, Bob and Carol to these 
anonymous identities.

\paragraph{Privacy}

%Say that Alice, Bob and Carol have one inbox each, similarly as in the email 
%system or Signal.
%Eve monitors the network on a national level.
%Now Eve can see one message originating from Alice, going to a server beyond 
%Eve's reach, and soon two equally-sized messages return from the server 
%near-simultaneously to Bob and Carol\footnote{%
%  Or equivalently, Eve observes where these messages end up in the storage and 
%  later observes Bob and Carol fetching these messages.
%}.
%(As we pointed out above, this is what is called a time-correlation attack.)
%This is what happens when Alice, Bob and Carol do not have perfect anonymity, 
%e.g.\ by using Tor or simply using Signal without any anonymizing service.
%Eve can relate Alice, Bob and Carol to each other.
%Now let us try to make this more difficult for Eve.
%
Say that Alice, Bob and Carol can access the storage system for messages 
anonymously, i.e.\ Eve can only observe when Alice, Bob and Carol does 
something --- but not what they do --- and when something happens in the 
storage system --- but not who does it.
Despite this anonymity, Eve can still do a correlation attack.
For example, Eve can temporarily detain Bob (or turn off his network 
connection) and observe the change in the distribution of reads from the 
storage.
In the push model, she will observe that someone stopped reading from one of 
the locations, i.e.\ Bob's location.
The same argument can be applied to the pull model case: when Eve detains Bob, 
she can observe a change in the probability distribution of reads from where 
Alice puts her messages.
In fact, even if several people share locations, this simply slow Eve down.

As was pointed out above (\cref{WhenAdversaryControlsNetwork}), the solution to 
this type of attack is to add noise to make these changes in distribution 
indistinguishable.
\citeauthor{PPACinPubFS} suggested that differential privacy\footnote{%
  Differential privacy guarantees that if we remove any \emph{one} data item, 
  the distribution will change below a given threshold.
} will probably be the best trade-off between privacy and efficiency.
In fact, parallel to the work of \citeauthor{PPACinPubFS}, 
\textcite{Vuvuzela,Alpenhorn} designed a protocol for one-to-one communication 
based on differential privacy.
There are still some limitations, e.g.\ everyone must be online and 
participating in the protocol all the time --- 24 hours every day.

%If one of the participants makes a mistake, then the regime's agents will have 
%a starting point to target.
%For example if a participant uses the same inbox for communication with all his 
%friends, and not only the participants in the plot against the regime, then one 
%of his other contacts might not be as concerned with staying anonymous.
%The consequence is that the regime can see the identity of someone sending 
%messages to an inbox of their interest.
%Then they can target this person and learn which friend owns the inbox of 
%interest.
%Then they can proceed to targeting one of the protest organizers.
%This type of attack will not work when the communication is according to the 
%pull model, since there the agency must attack each anonymous connection.
%%TODO: reviewer: explain what you mean by attack here
%




\subsection{Scheduling a Protest}
\label{Scheduling}

For the scheduling of a protest, there are in turn several problems that must 
be addressed.
From the organizer Alice's perspective, she wants to protect herself from being 
arrested for organizing a protest.
So Alice needs to protect herself from the possible participants, as one of 
them can be agent Eve of the intelligence services of the regime.
From the participant Bob's perspective, he wants to protect himself from being 
arrested for committing to participate in a protest.
So Bob needs to protect himself from the organizer and the other participants, 
as any of them can be Eve.

When organizing a protest, what Alice and Bob want to agree on is a time, 
a place and to ensure enough people will show up at that time and place.
Alice and Bob also want Eve to learn as little as possible of the plans so 
that she cannot curtail the protest.

\citet{EventsInvitations} presented a distributed protocol without the need of 
\iac{TTP}.
This protocol allows for different privacy settings:
\begin{itemize}
\item Alice discloses nothing to Bob, except the time and the place;
\item Alice discloses everything --- who the invitees are, who of those have 
  already committed etc.;
\item and every combination of settings in between.
\end{itemize}
Further, if Alice doesn't keep her promises, Bob has a proof which he can 
publish to everyone to show that Alice cheated.
Likewise, if Bob commits to attending, Alice has proof that Bob has done so and 
can show to everyone that Bob isn't present although he said he would.

One topic that must be explored is to adapt this protocol to introduce 
deniability.
Another interesting feature to include would be the choice of location.
With this, all participants can jointly agree on not only a time, but also 
a location.
This would help against the problem of announcing the location in advance.

% Event Invitations
\subsection{Inviting Participants}
\label{InvitingParticipants}

%For the scheduling of a protest, there are in turn several problems that must 
%be addressed.
%From the organizer Alice's perspective, she wants to protect herself from being 
%arrested for organizing a protest.
%So Alice needs to protect herself from the possible participants, as one of 
%them can be agent Eve of the intelligence services of the regime.
%From the participant Bob's perspective, he wants to protect himself from being 
%arrested for committing to participate in a protest.
%So Bob needs to protect himself from the organizer and the other participants, 
%as any of them can be Eve.
%
%When organizing a protest, what Alice and Bob want to agree on is a time, 
%a place and to ensure enough people will show up at that time and place.
%Alice and Bob also want Eve to learn as little as possible of the plans so 
%that she cannot curtail the protest.
%
%\citet{EventsInvitations} presented a distributed protocol for event 
%invitations.
%This protocol does not require any \ac{TTP} and allows for different privacy 
%settings:
%\begin{itemize}
%\item Alice discloses nothing to Bob, except the time and the place;
%\item Alice discloses everything --- who the invitees are, who of those have 
%  already committed etc.;
%\item and every combination of settings in between.
%\end{itemize}
%Further, if Alice breaks her promises, Bob has a proof which he can publish to 
%everyone to show that Alice cheated.
%Likewise, if Bob commits to attending, Alice has proof that Bob has done so and 
%can show to everyone that Bob isn't present although he said he would.
%This is useful input to a reputation system.
%
%One topic that must be explored is to adapt this protocol to introduce 
%deniability.
%Another interesting feature to include would be the choice of location.
%With this, all participants can jointly agree on not only a time, but also 
%a location.
%This would help against the problem of announcing the location in advance.

Alice wants to organize a demonstration.
A demonstration requires participants to have any effect, so Alice wants to 
invite Bob and Carol to participate in her demonstration.
There is one major issue here: trust.
Alice must ensure that Bob and Carol learn enough information to attend.
In case Bob or Carol collaborate (voluntarily or involuntarily) with Eve, 
neither should have enough information for Eve to prevent the demonstration.
Bob and Carol on the other hand want to ensure that Alice is not 
(collaborating with) Eve and that the invitation is not an attempt to trick 
them to \enquote{incriminate} themselves as conspirators against the regime.
\Textcite{EventsInvitations} has done some work in the direction of solving 
this problem with a fully decentralized scheme.

%There are some tasks that the organizers must accomplish prior to the protest 
%itself.
%For example, they must decide who are the most suitable candidates to attend 
%the event, how to let them know about the protest and what preliminary 
%information they should get.
%They must also decide whether invitees should learn about the attendance of 
%other invitees or not.
%Preferably, all this should be possible in a privacy-preserving
%fashion. %TODO: reviewer: unclear

In a centralized scheme (like Facebook), there is a third party (Facebook) who 
must be trusted to keep secrets and adhere to the protocol.
In contrast, with a decentralized protocol
Alice, Bob and Carol need only trust themselves --- not even each other.
This means that there is no central authority who has all invitation data that 
Eve can compel to give it up.

Realizing this rather standard feature of \acp{OSN} in a decentralized manner is not 
trivial, because there is no trusted third party for both organizers and 
invitees to rely on.
In this situation, Alice, Bob and Carol depend only on themselves.
This means that the protocol must guarantee fairness to all parties involved, 
e.g.\ that Bob can verify that the invitation he received was actually sent by 
Alice.
Moreover, the protocol must also provide privacy settings to protect personal 
information, such as the identities of Bob and Carol.
E.g.\ Alice can restrict that only the invited participants learn how many 
others have been invited, and only after Bob has agreed and formally committed 
to attend the event, he can learn the identities of other invited protesters 
(Carol).

The challenge of implementing this feature without a trusted third party 
becomes greater when Alice want to allow different types of information about 
the event to be shared with different groups of participants, e.g.\ Alice might 
trust Bob less than Carol.
The difficulty is Bob and Carol should be able to verify the results to detect 
if Alice tries to cheat.

In the scheme by \textcite{EventsInvitations}, they describe and formalize the 
security and privacy properties outlined above.
More specifically, Alice is able to configure who can learn the identities of 
the invited or attending participants, or a more restrictive version where only 
the number of invitees or attendees are revealed.
There is also an attendee-only property that guarantees exclusive access to 
some data, e.g.\ the location of the demonstration.
These properties are accomplished using several simple primitives:
storage-location indirection\footnote{%
  This means that Alice stores a secret in location \(A\), then she stores this
  location encrypted in location \(B\).
  She can reveal \(B\) to Bob and Carol, but they cannot find \(A\) without the
  decryption key.
  Later, Alice might give Bob or Carol the key and they can find \(A\).
},
controlled ciphertext-inference and
a commit-disclose protocol.
Storage-location indirection allows Alice to control not only who can read the 
contents of some encrypted data, but also who can access the ciphertext.
Alice can use controlled ciphertext-inference to allow for controlled 
information leaks.
E.g.\ Alice can use this to reveal the number of invitees but to keep their 
identities secret\footnote{%
  The identities are encrypted, but if Bob and Carol count the ciphertexts they
  learn the number of identities, e.g.\ the number of invitees or attendees.
}.
Finally, the commit-disclose protocol can make some secret available for only 
those invitees who committed to attend the demonstration while, at the same 
time, detect any misbehaving party.
If Alice tries to cheat Bob, then this protocol provides Bob with data 
that he can show to Carol to prove that Alice tries to cheat.
Consequently, Alice, Bob and Carol can use this to build a reputation for 
Alice.


\subsection{Having a Decentralized Account}
\label{passwd}
In general, there are some more properties that are relevant for these systems.
Some secure authentication schemes allows for \ac{DoS} attacks against the 
proper account holder.
Although it prevents the attacker from gaining access, it also prevents the 
authentic user:
e.g.\ what should happen if multiple users try to access an account at the same 
time?
If the intelligence services can lock the activists out of their accounts and 
thus forcing them to resolve to less secure means of communication, then the 
intelligence services have won.
\citet{P2PPasswords} developed mechanisms towards solving this problem.


