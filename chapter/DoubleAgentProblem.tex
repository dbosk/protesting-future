Consider the following scenario.
We have an oppressive regime and its oppressed opposition.
The opposition's goal is a protest against the regime's totalitarian authority.
A big-enough protest will eventually lead to a change in government and the 
formation of a democracy.
The regime's goal is to oppress the opposition so that they cannot ever reach 
a big-enough protest to show the majority's dissatisfaction with the regime.

The double agent problem is the problem of one of the regime's agents 
infiltrating the opposition by acting as if part of the opposition.
The reverse is of course also true, the opposition may have an agent in the 
government.

We cannot solve this problem, however, we might be able to reduce the damage.
If the double agent acts perfectly we cannot detect him.
However, it is very difficult to play this perfectly.
This means that in practice he can eventually be detected, but he will cause 
some damage until detected.
One aim of the technologies discussed is to reduce the damage that the double 
agent can do.

Throughout this chapter we will focus on the first version of the problem, 
i.e.\ that the regime tries to infiltrate the opposition.
The reason is that this is a more interesting since the adversary is stronger 
when having the upper hand.
However, before we continue our treatment we must consider a related problem 
which is introduced by the decentralization used in our setting.
