\subsection{Inviting Participants}
\label{InvitingParticipants}

%For the scheduling of a protest, there are in turn several problems that must 
%be addressed.
%From the organizer Alice's perspective, she wants to protect herself from being 
%arrested for organizing a protest.
%So Alice needs to protect herself from the possible participants, as one of 
%them can be agent Eve of the intelligence services of the regime.
%From the participant Bob's perspective, he wants to protect himself from being 
%arrested for committing to participate in a protest.
%So Bob needs to protect himself from the organizer and the other participants, 
%as any of them can be Eve.
%
%When organizing a protest, what Alice and Bob want to agree on is a time, 
%a place and to ensure enough people will show up at that time and place.
%Alice and Bob also want Eve to learn as little as possible of the plans so 
%that she cannot curtail the protest.
%
%\citet{EventsInvitations} presented a distributed protocol for event 
%invitations.
%This protocol does not require any \ac{TTP} and allows for different privacy 
%settings:
%\begin{itemize}
%\item Alice discloses nothing to Bob, except the time and the place;
%\item Alice discloses everything --- who the invitees are, who of those have 
%  already committed etc.;
%\item and every combination of settings in between.
%\end{itemize}
%Further, if Alice breaks her promises, Bob has a proof which he can publish to 
%everyone to show that Alice cheated.
%Likewise, if Bob commits to attending, Alice has proof that Bob has done so and 
%can show to everyone that Bob isn't present although he said he would.
%This is useful input to a reputation system.
%
%One topic that must be explored is to adapt this protocol to introduce 
%deniability.
%Another interesting feature to include would be the choice of location.
%With this, all participants can jointly agree on not only a time, but also 
%a location.
%This would help against the problem of announcing the location in advance.

Alice wants to organize a demonstration.
A demonstration requires participants to have any effect, so Alice wants to 
invite Bob and Carol to participate in her demonstration.
There is one major issue here: trust.
Alice must ensure that Bob and Carol learn enough information to attend.
In case Bob or Carol collaborate (voluntarily or involuntarily) with Eve, 
neither should have enough information for Eve to prevent the demonstration.
Bob and Carol on the other hand want to ensure that Alice is not 
(collaborating with) Eve and that the invitation is not an attempt to trick 
them to \enquote{incriminate} themselves as conspirators against the regime.
\Textcite{EventsInvitations} has done some work in the direction of solving 
this problem with a fully decentralized scheme.

%There are some tasks that the organizers must accomplish prior to the protest 
%itself.
%For example, they must decide who are the most suitable candidates to attend 
%the event, how to let them know about the protest and what preliminary 
%information they should get.
%They must also decide whether invitees should learn about the attendance of 
%other invitees or not.
%Preferably, all this should be possible in a privacy-preserving
%fashion. %TODO: reviewer: unclear

In a centralized scheme (like Facebook), there is a third party (Facebook) who 
must be trusted to keep secrets and adhere to the protocol.
In contrast, with a decentralized protocol
Alice, Bob and Carol need only trust themselves --- not even each other.
This means that there is no central authority who has all invitation data that 
Eve can compel to give it up.

Realizing this rather standard feature of \acp{OSN} in a decentralized manner is not 
trivial, because there is no trusted third party for both organizers and 
invitees to rely on.
In this situation, Alice, Bob and Carol depend only on themselves.
This means that the protocol must guarantee fairness to all parties involved, 
e.g.\ that Bob can verify that the invitation he received was actually sent by 
Alice.
Moreover, the protocol must also provide privacy settings to protect personal 
information, such as the identities of Bob and Carol.
E.g.\ Alice can restrict that only the invited participants learn how many 
others have been invited, and only after Bob has agreed and formally committed 
to attend the event, he can learn the identities of other invited protesters 
(Carol).

The challenge of implementing this feature without a trusted third party 
becomes greater when Alice want to allow different types of information about 
the event to be shared with different groups of participants, e.g.\ Alice might 
trust Bob less than Carol.
The difficulty is Bob and Carol should be able to verify the results to detect 
if Alice tries to cheat.

In the scheme by \textcite{EventsInvitations}, they describe and formalize the 
security and privacy properties outlined above.
More specifically, Alice is able to configure who can learn the identities of 
the invited or attending participants, or a more restrictive version where only 
the number of invitees or attendees are revealed.
There is also an attendee-only property that guarantees exclusive access to 
some data, e.g.\ the location of the demonstration.
These properties are accomplished using several simple primitives:
storage-location indirection\footnote{%
  This means that Alice stores a secret in location \(A\), then she stores this
  location encrypted in location \(B\).
  She can reveal \(B\) to Bob and Carol, but they cannot find \(A\) without the
  decryption key.
  Later, Alice might give Bob or Carol the key and they can find \(A\).
},
controlled ciphertext-inference and
a commit-disclose protocol.
Storage-location indirection allows Alice to control not only who can read the 
contents of some encrypted data, but also who can access the ciphertext.
Alice can use controlled ciphertext-inference to allow for controlled 
information leaks.
E.g.\ Alice can use this to reveal the number of invitees but to keep their 
identities secret\footnote{%
  The identities are encrypted, but if Bob and Carol count the ciphertexts they
  learn the number of identities, e.g.\ the number of invitees or attendees.
}.
Finally, the commit-disclose protocol can make some secret available for only 
those invitees who committed to attend the demonstration while, at the same 
time, detect any misbehaving party.
If Alice tries to cheat Bob, then this protocol provides Bob with data 
that he can show to Carol to prove that Alice tries to cheat.
Consequently, Alice, Bob and Carol can use this to build a reputation for 
Alice.
