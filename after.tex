\section{After a Protest}
\label{AfterProtest}

One problem in real-world demonstrations that has not yet been entirely solved 
is the crowd-counting problem, i.e.\ the verification of the participation in 
a demonstration.
After many demonstrations the count by police and that by the organizers 
differ, in some instances the difference can be hundreds of thousands.
There are numerous examples, e.g.\ the demonstrations in South 
Korea~\cite{2016DemonstrationsInSeoul} or the women's marches in the 
US~\cite{2017WomensMarchesInUS}, where there is difficulty in establishing the 
actual number of participants.
The methods for counting the crowds vary.
There are some techniques using computer vision found in the research 
literature, e.g.\ by~\textcite{CVCrowdCounting}.
These require images from the demonstration and provide no way to verify the 
authenticity of the count (except re-running the algorithm on one's own input 
data).
However, as is illustrated by 
\textcite{2016DemonstrationsInSeoul,2017WomensMarchesInUS}, the methods used in 
practice are manual and prone to errors.

The main goal for Alice after a protest is to provide verifiable data.
For example, how can she ensure that photos from a demonstration are authentic?
We can probably recognize the place the photo is portraying, however, this 
might just as well be a reconstruction or entirely computer generated.
We cannot trust the meta-data of the photo, such as time-stamps of the file, 
because these can easily be manipulated.
So the only thing we can say for sure is that the photo was taken at the latest 
at the time of publication.
Now, if we cannot trust these photos, how can we determine the number of 
participants of a demonstration?
Many of the techniques require photos for doing this.
Furthermore, as is shown by 
\textcite{2016DemonstrationsInSeoul,2017WomensMarchesInUS}, the numbers 
determined by these techniques have rather large margins.

Alice also has another problem.
Assume that she can provide this verifiable data.
Alice and Bob do not want that this data ties them to the demonstration, 
because then Eve can use this data to find out about Alice and Bob and arrest 
them.
Thus, what Alice and Bob need is a system which can provide data authenticity 
and user privacy, i.e.\ that the data can be correctly tied to the 
demonstration without outing Alice and Bob as supporters.

%\item the possibilities for the organizers to verify the participation and 
%use it as feedback into a reputation system;
  
\subsection{Data Authenticity}
\label{DataAuthenticity}

\citet{PPACforPubFS} presents work done in the area of privacy-preserving 
access control in distributed storage systems.
This is important since it outlines some possibilities and limits for such 
systems.
The problem in this scenario is that we don't want to be identifiable as 
a demonstrator, as this might yield repercussions.
Thus we can to share the data anonymously, no one can tell with whom we've 
shared what or if they've read it.
However, we still want to verify authenticity as it might otherwise be the 
regime spreading disinformation.

In this case it is also not straight forward to just apply the techniques in 
\cite{OTPKX} to achieve deniability.



\subsection{Verifying a Demonstration}
\label{ProtestVerif}

One problem in physical protesting that has not yet been solved satisfactorily 
is the verification of the participation in a protest.
After many protests the counts of demonstrators of police and the organizers 
differ, in some instances the difference can be up to hundreds of 
thousands~\cite{ExampleProtestCount}.

Many of the current methods to count the participants are based on computer 
vision, i.e.\ object recognition through image analysis.
A demonstration is very similar to voting, both are many individuals expressing 
their opinion.
Hence it is desirable to have similar properties for verifying the 
participation in a protest.
Instead of votes we could have participation proofs and the following 
properties:
\begin{properties}
  \item\label{VerifEligibility} Everyone can verify that every participation 
    proof is correct.
  \item\label{VerifIndividual} Every participant can verify that its 
    participation proof is included in the count.
  \item\label{VerifUniversal} Everyone can verify that the count is according 
    to the participation proofs.
\end{properties}
This is difficult to accomplish with computer vision methods.
The properties outlined above are indeed desirable, e.g.\ then the \ac{UN} can 
verify protests happening in a country and the country cannot deny it, thus the 
\ac{UN} can apply pressure if needed.

Additionally, in voting, the cast votes are not linkable to the identity of the 
casters.
Thus it is not a problem to reveal the identites of those who participated in 
the vote, since they could have voted for any alternative.
We would also like to have the corresponding property in verifying the 
protest.
The very nature of a protest is different though, we do not have any choice: if 
we participate we support the cause of the protest.
Consequently we want to verify the participation of a protest without 
identifying individuals who participated.
Otherwise the regime's agents can identity all the participants and simply 
\enquote{make them disappear}.

For real-world protests we need to bind participants to the same physical 
location at a reasonably similar time, i.e.\ within the area and duration of 
the protest.
\citet{PROPS} developed a decentralized \ac{LPS} which provides a participant 
with a verifiable proof of having been at a location at a certain time, 
something we call \iac{LP}.
It is decentralized because there is no central authority that verifies the 
location, instead peers act as witnesses.
Then a third-party can verify the authenticity of the \ac{LP}, by verifying the 
witnesses signatures, and can thus be sure that the person has indeed been in 
the location.

Bosk, Gambs and Buchegger are currently exploring (work in progress) the 
possibility of turning such \iac{LPS} into a system for verifying the 
participation in protests.
The idea is that each participant generates \iac{LP} during the protest, where 
(some of) the other protesters act as witnesses, then the \acp{LP} can be used 
to compute the participation count.


