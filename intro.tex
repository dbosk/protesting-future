\section{Introduction}
\label{Introduction}

Online technologies have become an essential part of the lives of millions of people 
all over the world. The increasing prevalence of technology has helped society reach 
the 21\textsuperscript{st} century with a high standard of living, for example, real-time 
secure communication was a science-fiction notion a century ago. However, the success 
of such development has come with some trade offs, for example in data collection, 
used by private and public entities not only to provide new services but also to 
endanger the privacy of citizens, and sometimes the safety, usually in oppressive 
regimes.

Among the technologies developed to date, \acp{OSN} are a popular application in 
the last couple of decades. Computation power and network communication are combined 
to make social interactions between people possible at any time and in any place 
lessening political, economical and geographical boundaries.

Many of the services 
offering a \ac{OSN} are run in a centralized manner --- the providers of the service 
act as a communication channel between the users of the service. Such structure 
allows the providers to oversee a large, if not all, portion of the data exchanged 
between the users. In the case of \acp{OSN}, much of it is of personal and sensititive 
kind, for example, posting a picture in the network may reveal the physical geolocation 
as it can be embedded in the meta-data of the image.

Moreover, the large collection of data in these networks makes them an ideal target 
for attackers such as competitors or even governmental agencies. For example, in 
the recent years, intelligence and security agencies of some countries have targeted 
these services to gain personal information about their citizens, enemies and even 
allies~\cite{Prism}.

We can conclude that we need strong privacy properties for online technologies 
to protect citizens in all countries of the world.
One interesting branch of research is decentralized \acp{PET}.
Decentralized solutions yield provider independence and censorship resistance.
Privacy-preserving solutions provide data protection by prevention, e.g.\ by 
cryptographic means rather than that some organization must maintain a secure 
system.

This chapter focuses on technical tools that can be useful for protesting.
We focus on privacy-preserving tools that exist in the research literature of 
the security and privacy field but have not yet seen widespread use in practice 
and some future research directions.
Protesting can be done physically, as we have known protests traditionally, 
they can also be done purely online in the form of \enquote{petitions} or, more 
generally, by expressing support for an expressed opinion.
The petitions can be seen as the problem of electronic voting, we can transform 
the petition into a vote.
This would not be straight-forward and would include interesting research 
problems to be solved, however, we will not discuss this in the chapter.
The more general case of expressing support for an expressed opinion is 
something that can be done using social media, which is more related to the 
\ac{OSN} mechanisms that we will discuss.
We will focus on the latter, and especially its use related to physical 
protests.

We categorize the topics we will cover as useful before, under or after 
a protest.
The before stage treats problems related to how to organize a public protest, 
this is discussed in \cref{BeforeProtest}.
The problems in the during the protest, e.g.\ how the organizers and 
participants can communicate securely within the group or to the outside world, 
is treated in \cref{DuringProtest}.
Finally, in the after stage we are interested in achieving different 
authenticity and verifiability properties, e.g.\ possibly following up an event 
by verifying the participation and computing verifiable statistics, such as how 
many participants and in what area.
This is treated in \cref{AfterProtest}.
Some tools are of course useful in more than one stage, so the categorization 
is not strict in that sense.
Furthermore, some tools that are useful afterwards requires us to perform some 
action during the protest, we will cover such a topic as a unit and simply 
point out what must be performed during and what must be done after the 
protest.

Before we begin our treatment of what was just outlined, there are two 
fundamental problems that we must discuss.
These are the problem of double agents and the problem of Sybil attacks.

\subsection{The Double Agent Problem}
\label{DoubleAgentProblem}

Consider the following scenario.
We have an oppressive regime and its oppressed opposition.
The opposition's goal is a protest against the regime's totalitarian authority.
A big-enough protest will eventually lead to a change in government and the 
formation of a democracy.
The regime's goal is to oppress the opposition so that they cannot ever reach 
a big-enough protest to show the majority's dissatisfaction with the regime.

The double agent problem is the problem of one of the regime's agents 
infiltrating the opposition by acting as if part of the opposition.
The reverse is of course also true, the opposition may have an agent in the 
government.

We cannot solve this problem, however, we might be able to reduce the damage.
If the double agent acts perfectly we cannot detect him.
However, it is very difficult to play this perfectly.
This means that in practice he can eventually be detected, but he will cause 
some damage until detected.
One aim of the technologies discussed is to reduce the damage that the double 
agent can do.

Throughout this chapter we will focus on the first version of the problem, 
i.e.\ that the regime tries to infiltrate the opposition.
The reason is that this is a more interesting since the adversary is stronger 
when having the upper hand.
However, before we continue our treatment we must consider a related problem 
which is introduced by the decentralization used in our setting.


\subsection{The Sybil Attack}
\label{SybilAttacks}

We have to deal with the problem of \emph{one} person being counted twice by 
signing the petition using \emph{two identities}.
This is generally known as the problem of Sybil attacks, and it has been proven 
impossible to solve without \emph{logically} central control of the creation of 
identities~\cite{SybilAttack}.

\dots


