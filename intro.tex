\section{Introduction}
\label{Introduction}
%Online technologies have become an essential part of millions of people all over the world. 
The rapid development of technology in the latter half of the
20\textsuperscript{th} century and its increasing prevalence in
everyday life has helped large parts of the world to reach the
21\textsuperscript{st} century with a means of having real-time secure
communications.  However, the success of such development has come
with some trade offs, for example, in data collection. Better storage
technologies have allowed for longer data retention policies for both
the private and public sectors, not only providing new and better
services and combat crime, but also compromising the privacy of
citizens, and sometimes their safety in oppressive regimes.

Among these technological advances, \acp{OSN} stand out as a popular
computer-mediated tool allowing people and other entities to interact
by sharing and exchanging information of any kind. Computation power
and network communication are combined to make social interactions
between people possible at any time and in any place lessening
political, economical and geographical boundaries. These social media
are increasingly used for political activism ranging from showing
one's leanings by liking something to actual organization and support for
protests.

Many \acp{OSN} are run in a centralized manner --- the service
provider acts as a communication channel between the users of the
\ac{OSN}. Such structure allows providers to oversee a large portion
of the data, if not all, exchanged between the users. Bearing in mind
that in the case of \acp{OSN}, much of it is of personal and sensitive
kind, for example, posting a picture in the network may reveal the
physical geolocation as this information can be embedded in the
meta-data of the image. This has proved to be problematic for
political activisim in several ways. A centralized ownership and
control make it easier to shut down (cf. the banning of Twitter in
Turkey for several days). Governments can subpoena 
information from the service provider.  Moreover, the massive
collection of data in these networks makes them an ideal target for
attackers such as competitors or even governmental agencies. For
example, in the recent years, intelligence and security agencies of
some countries have targeted these services to gain personal
information about their citizens, enemies and even
allies~\cite{Prism}. Since a centralized system can log not just data
the users upload but also meta-information about their behavior, such
as online times, whom they communicate with, their location and social
ties, there is a wealth of information to connect a person to a cause.

While we acknowledge the benefits of such technological advances like
\acp{OSN}, we also point out the costs to personal privacy and
advocate for the need to develop privacy-enhancing technologies that
can co-exist with these technologies. For example, decentralized
solutions try to achieve provider independence and, in some cases,
they also offer censorship resistance. Privacy-preserving solutions
provide data protection by prevention, for example, by means of
cryptographic techniques an organization could enforce certain
policies instead of relying on the security of the system and its
maintenance.

Besides providing technological support to the conventional and
long-established form of protesting physically, online technologies
have also opened the possibility to alternative ways, such as virtual
\enquote{petitions}, or in general, expressing support for an opinion
in the form of an encouraging comment or simply affirmation.

% The more general case of expressing support for an opinion can be done in a hybrid 
% manner using both physical contact and social media, such as \acp{OSN}. Petitions 
% can be considered a particular case of electronic voting, a field of its own that 
% is out of the scope of this chapter. 

In this chapter we focus on describing some privacy-enhancing tools in
the context of \acp{OSN} that we believe can be useful in a protest
and that have not yet seen a widespread use in practice. Although a
protest itself relies mainly on the traditional physical act of
gathering, we believe that it would benefit from some of the
developments originated in the fields of information security and
privacy.

We categorize the topics of this chapter in a time-event manner with respect to the 
stages of a protest, namely: before, during and after.
\begin{itemize}
    \item \textbf{Before}\\
    Organization, for example, decisions on the aim of the protest or the target audience that 
    is expected to participate in the protest.\\
    We address some of these issues in our scenario of \acp{OSN} in \cref{BeforeProtest}.

    \item \textbf{During}\\
    Communication during the protest, for example, 
    the organizers may need to get in touch with the press over the phone during 
    the protest.\\
    We discuss how these communications can be better protected in \cref{DuringProtest}.

    \item \textbf{After}\\
    Following up a protest by the organizers, not 
    only to assess their success but also to correct the flaws for the next time. 
    For example, the organizers may want to obtain reliable statistics on the number 
    of attendees per area.\\
    We discuss different authenticity and verifiability properties of use for this 
    stage in \cref{AfterProtest}.
\end{itemize}

Note that some of the tools and techniques we describe may be useful in more than 
one stage. Furthermore, there are some prerequisites for some of the tools that 
require either the protesters to perform some action in a stage prior to the one 
where the tool is actually used, for example, if an invited participant must confirm 
the attendance before receiving details about the location of the protest.

\subsection{TODO: Give me a nice title}
% TODO Frame the following two problems. How do they fit (and help) in our scenario?
Before we begin our treatment of what was just outlined, there are two 
fundamental problems that we must discuss.
These are the problem of double agents and the problem of Sybil attacks.

\subsubsection{The Double Agent Problem}
\label{DoubleAgentProblem}

Consider the following scenario.
We have an oppressive regime and its oppressed opposition.
The opposition's goal is a protest against the regime's totalitarian authority.
A big-enough protest will eventually lead to a change in government and the 
formation of a democracy.
The regime's goal is to oppress the opposition so that they cannot ever reach 
a big-enough protest to show the majority's dissatisfaction with the regime.

The double agent problem is the problem of one of the regime's agents 
infiltrating the opposition by acting as if part of the opposition.
The reverse is of course also true, the opposition may have an agent in the 
government.

We cannot solve this problem, however, we might be able to reduce the damage.
If the double agent acts perfectly we cannot detect him.
However, it is very difficult to play this perfectly.
This means that in practice he can eventually be detected, but he will cause 
some damage until detected.
One aim of the technologies discussed is to reduce the damage that the double 
agent can do.

Throughout this chapter we will focus on the first version of the problem, 
i.e.\ that the regime tries to infiltrate the opposition.
The reason is that this is a more interesting since the adversary is stronger 
when having the upper hand.
However, before we continue our treatment we must consider a related problem 
which is introduced by the decentralization used in our setting.


\subsubsection{The Sybil Attack}
\label{SybilAttacks}

We have to deal with the problem of \emph{one} person being counted twice by 
signing the petition using \emph{two identities}.
This is generally known as the problem of Sybil attacks, and it has been proven 
impossible to solve without \emph{logically} central control of the creation of 
identities~\cite{SybilAttack}.

\dots


